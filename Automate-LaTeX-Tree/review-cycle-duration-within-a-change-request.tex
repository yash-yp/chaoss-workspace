\hypertarget{review-cycle-duration-within-a-change-request}{%
\section{Review Cycle Duration within a Change
Request}\label{review-cycle-duration-within-a-change-request}}

Question: What is the duration of a review cycle within a single change
request?

\hypertarget{description}{%
\subsection{Description}\label{description}}

A change request is based on one or more review cycles. Within a review
cycle, one or more reviewers can provide feedback on a proposed
contribution. The duration of a review cycle, or the time between each
new iteration of the contribution, is the basis of this metric.

\hypertarget{objectives}{%
\subsection{Objectives}\label{objectives}}

This metric provides maintainers with insight on: Code review process
decay, as there are more iterations and review cycle durations increase.
Process bottlenecks resulting in long code review iterations. Abandoned
or semi-abandoned processes in the review cycles, where either the
maintainer or the submitter is slow in responding. Characteristics of
reviews that have different cyclic pattern lengths.

\hypertarget{implementation}{%
\subsection{Implementation}\label{implementation}}

Review Cycle Duration is measured as the time length of one review cycle
within a single change request. The duration can be calculated between:
The moment when each review cycle begins, defined as the point in time
when a change request is submitted or updated. The moment when each
review cycle ends, either because the change request was updated and
needs a new review or because it was accepted or rejected.

\hypertarget{filter}{%
\subsubsection{Filter}\label{filter}}

Average or Median Duration, optionally filtered or grouped by: Number of
people involved in review Number of comments in review Edits made to a
change request Project or program Organization making the change request
Time the change request was submitted Developers who contributed to a
change request Change request Number of review cycle on a change request
(e.g., filter by first, second, \ldots{} round)

\hypertarget{visualizations}{%
\subsubsection{Visualizations}\label{visualizations}}

\hypertarget{tools-providing-the-metric}{%
\subsubsection{Tools Providing the
Metric}\label{tools-providing-the-metric}}

\hypertarget{references}{%
\subsection{References}\label{references}}

Example of data that could be used to develop the metric:
\url{https://gerrit.wikimedia.org/r/c/mediawiki/core/+/194071}
