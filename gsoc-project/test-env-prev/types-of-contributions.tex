\hypertarget{types-of-contributions}{%
\section{Types of Contributions}\label{types-of-contributions}}

Question: What types of contributions are being made?

\hypertarget{description}{%
\subsection{Description}\label{description}}

Multiple, varied contributions make an open source project healthy. Many
projects have community members who do not write code but contribute in
equally valuable ways by managing the community, triaging bugs,
evangelizing the project, supporting users, or helping in other ways.

\hypertarget{objectives}{%
\subsection{Objectives}\label{objectives}}

A variety of contribution types can demonstrate that a project is mature
and well-rounded with sufficient activity to support all aspects of the
project, and enable paths to leadership that are supportive of a variety
of contribution types and people with varying expertise beyond coding.

\hypertarget{implementation}{%
\subsection{Implementation}\label{implementation}}

How contributions are defined, quantified, tracked and made public is a
challenging question. Answers may be unique to each project and the
following suggestions are a starting point. As a general guideline, it
is difficult to compare different contribution types with each other and
they might better be recognized independently.

\begin{itemize}
\tightlist
\item
  The following list can help with identifying contribution types:

  \begin{itemize}
  \tightlist
  \item
    Writing Code
  \item
    Reviewing Code
  \item
    Bug Triaging
  \item
    Quality Assurance and Testing
  \item
    Security-Related Activities
  \item
    Localization/L10N and Translation
  \item
    Event Organization
  \item
    Documentation Authorship
  \item
    Community Building and Management
  \item
    Teaching and Tutorial Building
  \item
    Troubleshooting and Support
  \item
    Creative Work and Design
  \item
    User Interface, User Experience, and Accessibility
  \item
    Social Media Management
  \item
    User Support and Answering Questions
  \item
    Writing Articles
  \item
    Public Relations - Interviews with Technical Press
  \item
    Speaking at Events
  \item
    Marketing and Campaign Advocacy
  \item
    Website Development
  \item
    Legal Council
  \item
    Financial Management
  \end{itemize}
\end{itemize}

\hypertarget{data-collection-strategies}{%
\subsubsection{Data Collection
Strategies}\label{data-collection-strategies}}

\begin{itemize}
\item
  \textbf{Interview or Survey:} Ask community members to recognize
  others for their contributions to recognize contribution types that
  have previously not been considered.

  \begin{itemize}
  \tightlist
  \item
    Who in the project would you like to recognize for their
    contributions? What did they contribute?
  \end{itemize}
\item
  \textbf{Observe project:} Identify and recognize leads of different
  parts of the project.

  \begin{itemize}
  \tightlist
  \item
    What leaders are listed on the project website or in a repository?
  \end{itemize}
\item
  \textbf{Capture Non-code Contributions:} Track contributions through a
  dedicated system, e.g., an issue tracker.

  \begin{itemize}
  \tightlist
  \item
    Logging can include custom information a project wants to know about
    non-code contributions to recognize efforts.
  \item
    Proxy contributions through communication channel activity. For
    example, If quality assurance members (QA) have their own mailing
    list, then activity around QA contributions can be measured by proxy
    from mailing list activity.
  \end{itemize}
\item
  \textbf{Collect Trace Data:} Measure contributions through
  collaboration tool log data.

  \begin{itemize}
  \tightlist
  \item
    For example, code contributions can be counted from a source code
    repository, wiki contributions can be counted from a wiki edit
    history, and email messages can be counted from an email archive
  \end{itemize}
\item
  \textbf{Automate Classification:} Train an artificial intelligence
  (AI) bot to detect and classify contributions.

  \begin{itemize}
  \tightlist
  \item
    An AI bot can assist in categorizing contributions, for example,
    help requests vs. support provided, or feature request vs. bug
    reporting, especially if these are all done in the same issue
    tracker.
  \end{itemize}
\end{itemize}

\emph{Other considerations:}

\begin{itemize}
\tightlist
\item
  Especially with automated reports, allow community members to opt-out
  and not appear on the contribution reports.
\item
  Acknowledge imperfect capture of contribution types and be explicit
  about what types of contributions are included.
\item
  As a project evolves, methods for collecting types of contributions
  will need to adapt. For example, when an internationalization library
  is exchanged, project activity around localization conceivably
  produces different metrics before and after the change.
\item
  Account for activity from bots when mining contribution types at large
  scale.
\end{itemize}

\hypertarget{references}{%
\subsection{References}\label{references}}

\begin{itemize}
\tightlist
\item
  \url{https://medium.com/@sunnydeveloper/revisiting-the-word-recognition-in-foss-and-the-dream-of-open-credentials-d15385d49447}
\item
  \url{https://24pullrequests.com/contributing}
\item
  \url{https://smartbear.com/blog/test-and-monitor/14-ways-to-contribute-to-open-source-without-being/}
\item
  \url{https://wiki.openstack.org/wiki/AUCRecognition}
\item
  \url{https://www.drupal.org/drupalorg/blog/a-guide-to-issue-credits-and-the-drupal.org-marketplace}
\end{itemize}
